\hyphenation{
le-gem
pone
mihi
dó-mine
viam
ius-ti-fi-ca-ti-ó-num
tuá-rum
exquí-ram
eam
sem-per
mihi
intel-léc-tum
scru-tá-bor
le-gem
tuam
custó-diam
il-lam
toto
corde
meo
de-duc
sé-mi-tam
man-da-tó-rum
tuó-rum
quia
i-psam
vó-lui
inclína
cor
meum
testi-mó-nia
tua
non
ava-rí-tiam
avérte
ócu-los
meos
ví-de-ant
vani-tá-tem
via
tua
viví-fica
stá-tue
servo
tuo
eló-quium
tuum
timóre
tuo
mputa
oppró-brium
meum
quod
suspi-cá-tus
sum
quia
iudí-cia
tua
iucúnda
ecce
con-cu-pívi
man-dáta
tua
æqui-táte
tua
viví-fi-ca-vé-niat
su-per
mise-ri-cór-dia
tua
dó-mine
salu-táre
tuum
secún-dum
eló-quium
tuum
respon-débo
expro-brán-ti-bus
mihi
ver-bum
quia
spe-rávi
ser-mó-ni-bus
tuis
áu-fe-ras
ore
meo
ver-bum
veri-tá-tis
usque-quá-que
quia
iudí-ciis
tuis
super-spe-rávi
custó-diam
le-gem
tuam
sem-per
sǽ-cu-lum
sǽ-cu-lum
sǽ-culi
amb-u-lá-bam
lati-tú-dine
quia
man-dáta
tua
exqui-sívi
loqué-bar
testi-mó-niis
tuis
con-spéctu
re-gum
non
con-fun-dé-bar
medi-tá-bar
man-dá-tis
tuis
quæ
diléxi
levávi
ma-nus
meas
man-dáta
tua
quæ
diléxi
exer-cé-bar
ius-ti-fi-ca-ti-ó-ni-bus
tui-sme-mor
esto
verbi
tui
servo
tuo
quo
mihi
spem
dedí-sti
hæc
con-so-láta
est
humi-li-táte
mea
quia
eló-quium
tuum
vivi-fi-cá-vit
supérbi
iní-que
agé-bant
usque-quá-que
lege
au-tem
tua
non
decli-návi
me-mor
fui
iudi-ci-ó-rum
tuó-rum
sǽ-culo
dó-mine
con-so-lá-tus
sum
deféc-tio
té-nuit
pro
pec-ca-tó-ri-bus
dere-lin-quén-ti-bus
le-gem
tuam
can-tá-bi-les
mihi
e-rant
ius-ti-fi-ca-ti-ó-nes
tuæ
loco
per-e-gri-na-ti-ó-nis
meæ
me-mor
fui
nocte
nó-mi-nis
tui
dó-mine
custo-dívi
le-gem
tuam
hæc
facta
est
mihi
quia
ius-ti-fi-ca-ti-ó-nes
tuas
exqui-sí-vi-do-mi-nica
secunda
pas-si-o-nis
seu
pal-mis
domi-nica
pas-si-o-nis
du-plex
clas-sis
ter-tiam
vspace
hym-nus
psal-mus
ant
capi-tu-lum
phi-lipp
fra-tres
hoc
e-nim
sen-títe
vo-bis
quod
chri-sto
iesu
qui
cum
forma
dei
es-set
non
rapí-nam
arbi-trá-tus
est
esse
æquá-lem
deo
sed
semet-í-psum
exin-a-ní-vit
for-mam
servi
accí-pi-ens
simi-li-tú-di-nem
hó-mi-num
fac-tus
há-bitu
invén-tus
homo
deo
grá-tias
ver-si-culi
varia
varia
deinde
dici-tur
ver-si-culi
oré-mus
ora-tio
omní-po-tens
sem-pi-térne
deus
qui
humáno
gé-neri
imi-tán-dum
humi-li-tá-tis
exém-plum
sal-va-tó-rem
nos-trum
car-nem
sú-mere
cru-cem
sub-íre
fecí-sti
con-céde
pro-pí-tius
pati-én-tiæ
ipsíus
habére
docu-ménta
resur-rec-ti-ó-nis
con-sór-tia
mere-á-mur
per
eún-dem
dó-mi-num
nos-trum
ie-sum
chri-stum
fí-lium
tuum
qui
te-cum
vi-vit
re-gnat
uni-táte
spí-ri-tus
sancti
deus
per
óm-nia
sǽ-cula
sæ-cu-ló-rum
a-men
post
ora-ti-o-nem
addi-tur
ver-si-culi
ver-si-culi
sim-plex
ante
mis-sam
pon-ti-fi-ca-lem
dici-tur
ver-si-culi
pon-ti-fi-ca-lis
sol-emni
pro-ces-si-one
hono-rem
chri-sti
re-gis
pro-ces-si-o-nem
mis-sam
sexta
}
